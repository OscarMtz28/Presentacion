\documentclass{beamer}
\usepackage[spanish]{babel}
\usepackage[utf8]{inputenc}
\usepackage{graphicx}
\usepackage{hyperref}

\title{Arquitectura de Sistemas de Bases de Datos}
\subtitle{Componentes, Usuarios y Administración}
\author{Jose Alberto González }
\date{\today}
\institute{Universidad o Empresa}

\usetheme{Madrid} % Tema opcional (Berlin, Dresden, etc.)

\begin{document}

\begin{frame}
  \titlepage
\end{frame}

% --- INTRODUCCIÓN ---
\section{Introducción}
\begin{frame}{¿Qué cubriremos?}
  \begin{itemize}
    \item Arquitecturas de bases de datos (centralizada, paralela, distribuida).
    \item Tipos de usuarios y sus interacciones.
    \item Funciones clave del Administrador (DBA).
  \end{itemize}
\end{frame}

% --- ARQUITECTURAS ---
\section{Arquitecturas de Bases de Datos}
\begin{frame}{Arquitectura Centralizada}
  \begin{block}{Figura 1.3: Modelo Cliente-Servidor}
    \begin{itemize}
      \item Ejecución en un servidor con memoria compartida.
      \item Múltiples CPUs acceden a la misma memoria.
      \item Usado en sistemas tradicionales.
    \end{itemize}
    \end{block}
  \centering
  %\includegraphics[width=0.5\textwidth]{esquema_centralizado.png} % Reemplaza con tu imagen
\end{frame}

\begin{frame}{Arquitecturas Avanzadas}
  \begin{columns}
    \column{0.5\textwidth}
      \textbf{Bases de Datos Paralelas}
      \begin{itemize}
        \item Cluster de máquinas.
        \item Alto volumen de datos.
      \end{itemize}
    \column{0.5\textwidth}
      \textbf{Bases de Datos Distribuidas}
      \begin{itemize}
        \item Datos en ubicaciones geográficas distintas.
        \item Ejemplo: Sistemas globales.
      \end{itemize}
  \end{columns}
\end{frame}

% --- USUARIOS ---
\section{Tipos de Usuarios}
\begin{frame}{4 Tipos de Usuarios}
  \begin{enumerate}
    \item \textbf{Ingenuos (Naïve)}: Interfaz predefinida (formularios web/móviles).
    \item \textbf{Programadores}: Desarrollan aplicaciones con herramientas.
    \item \textbf{Avanzados (Sophisticated)}: Consultas directas (SQL, análisis de datos).
    \item \textbf{Administradores (DBA)}: Control centralizado.
  \end{enumerate}
\end{frame}

% --- DBA ---
\section{Rol del Administrador (DBA)}
\begin{frame}{Funciones del DBA}
  \begin{itemize}
    \item Definición y modificación de esquemas.
    \item Optimización de almacenamiento (índices, particiones).
    \item \textcolor{blue}{Gestión de permisos} (seguridad por autorizaciones).
    \item Mantenimiento: Backups, espacio en disco, monitoreo.
  \end{itemize}
\end{frame}

\begin{frame}{Importancia del DBA}
  \begin{alertblock}{¿Por qué es crítico?}
    Garantiza:
    \begin{itemize}
      \item \textbf{Disponibilidad}: Copias de seguridad y recuperación.
      \item \textbf{Rendimiento}: Ajuste de consultas y recursos.
      \item \textbf{Seguridad}: Control de accesos no autorizados.
    \end{itemize}
  \end{alertblock}
\end{frame}

% --- CONCLUSIÓN ---
\section{Conclusión}
\begin{frame}{Resumen}
  \begin{itemize}
    \item Arquitecturas evolucionan para escalar (centralizada $\rightarrow$ distribuida).
    \item Usuarios tienen necesidades técnicas distintas.
    \item El DBA es el "guardián" de los datos.
  \end{itemize}
\end{frame}

\begin{frame}{¿Preguntas?}
  \centering
  \Large ¡Gracias por su atención! \\
  \small Contacto: \href{mailto:tu@email.com}{tu@email.com}
\end{frame}

\end{document}