\documentclass{beamer}
\usetheme{Madrid}
\usecolortheme{seahorse}
\usepackage[spanish]{babel}
\usepackage[utf8]{inputenc}
\usepackage{graphicx}
\usepackage{booktabs}
\usepackage{array}
\usepackage{ragged2e}
\usepackage{tikz}
\usetikzlibrary{positioning}
\usetikzlibrary{shapes,arrows}

\title{Arquitectura de Sistemas de Bases de Datos}
\subtitle{Capítulos 1.7 - 1.8: Usuarios y Administración}
\author{}
\date{}

\begin{document}

\begin{frame}
\titlepage
\end{frame}

\section{1. Visión General de Arquitectura}
\begin{frame}{Arquitectura Unificada de Bases de Datos}
\begin{block}{Componentes interconectados}
Sistema que permite interacción de diferentes usuarios con los datos
\end{block}

\begin{columns}[T]
\column{0.5\textwidth}
\textbf{Elementos clave:}
\begin{itemize}
    \item Gestión de transacciones
    \item Control de concurrencia
    \item Mecanismos de recuperación
    \item Interfaces de usuario
\end{itemize}

\column{0.5\textwidth}
\textbf{Objetivos:}
\begin{itemize}
    \item Consistencia de datos
    \item Seguridad
    \item Escalabilidad
    \item Alto rendimiento
\end{itemize}
\end{columns}

\begin{center}
\begin{tikzpicture}[node distance=1.3cm]
\node (db) [draw, rectangle, fill=blue!20] {Base de Datos};
\node (app) [draw, rectangle, fill=green!20, above left of=db] {Aplicaciones};
\node (users) [draw, ellipse, fill=yellow!20, above of=app] {Usuarios};
\node (dba) [draw, rectangle, fill=red!20, above right of=db] {DBA};

\draw [->] (users) -- (app);
\draw [->] (app) -- (db);
\draw [->] (dba) -- (db);
\end{tikzpicture}
\end{center}
\end{frame}

\section{2. Tipos de Arquitecturas}
\begin{frame}{Modelos de Implementación}
\begin{columns}[T]
\column{0.33\textwidth}
\textbf{Centralizada}
\begin{itemize}
    \item Unico servidor/múltiples CPUs
    \item Memoria compartida
    \item \alert{Limitaciones:}
    \begin{itemize}
        \item Cuellos de botella
        \item Escalabilidad reducida
    \end{itemize}
    \item \textbf{Ejemplo:} Banco pequeño
\end{itemize}

\column{0.33\textwidth}
\textbf{Paralela}
\begin{itemize}
    \item Múltiples máquinas (mismo DC)
    \item Procesamiento masivo
    \item \textbf{Ventajas:}
    \begin{itemize}
        \item Alta velocidad
        \item Cálculos intensivos
    \end{itemize}
    \item \textbf{Uso:} Big Data, IA
\end{itemize}

\column{0.33\textwidth}
\textbf{Distribuida}
\begin{itemize}
    \item Servidores geográficamente dispersos
    \item \textbf{Ventajas:}
    \begin{itemize}
        \item Baja latencia
        \item Tolerancia a fallos
    \end{itemize}
    \item \textbf{Uso:} eCommerce global
\end{itemize}
\end{columns}
\end{frame}

\begin{frame}{Evolución de Arquitecturas de Aplicaciones}
\begin{columns}[T]
\column{0.5\textwidth}
\textbf{2 Niveles (Cliente-Servidor)}
\begin{itemize}
    \item Cliente $\leftrightarrow$ Servidor DB
    \item \alert{Problemas:}
    \begin{itemize}
        \item Sobrecarga servidor
        \item Vulnerabilidades seguridad
        \item Escalabilidad limitada
    \end{itemize}
\end{itemize}

\begin{center}
\begin{tikzpicture}[scale=1]
\node (client) [draw] {Cliente};
\node (server) [draw, right=.6cm of client] {Servidor DB};
\draw [<->] (client) -- (server);
\end{tikzpicture}
\end{center}

\column{0.5\textwidth}
\textbf{3 Niveles (Moderno)}
\begin{itemize}
    \item Cliente $\rightarrow$ Servidor Apps $\rightarrow$ DB
    \item \textbf{Ventajas:}
    \begin{itemize}
        \item Mayor seguridad
        \item Mejor rendimiento
        \item Escalabilidad mejorada
    \end{itemize}
\end{itemize}

\begin{center}
\begin{tikzpicture}[scale=0.4]
\node (client) [draw] {Cliente};
\node (app) [draw, right=.4cm of client] {Servidor Apps};
\node (db) [draw, below=.35cm of app] {Servidor DB};
\draw [->] (client) -- (app);
\draw [->] (app) -- (db);
\end{tikzpicture}
\end{center}
\end{columns}

\textbf{Ejemplo e-commerce:}
\begin{itemize}
    \item Usuario (navegador) $\rightarrow$ Servidor aplicaciones $\rightarrow$ Base de datos
\end{itemize}
\end{frame}

\section{3. Usuarios de Bases de Datos}
\begin{frame}{Tipos de Usuarios}
\begin{columns}[T]
\column{0.25\textwidth}
\textbf{Ingenuos}
\begin{itemize}
    \item Sin conocimiento técnico
    \item Interfaces predefinidas
    \item \textbf{Ejemplo:} 
    Estudiante que se inscribe vía web
\end{itemize}

\column{0.25\textwidth}
\textbf{Programadores}
\begin{itemize}
    \item Desarrollan aplicaciones
    \item Crean interfaces
    \item \textbf{Ejemplo:} 
    App e-commerce
\end{itemize}

\column{0.25\textwidth}
\textbf{Sofisticados}
\begin{itemize}
    \item Consultas directas (SQL)
    \item Herramientas analíticas
    \item \textbf{Ejemplo:} 
    Analista de negocios
\end{itemize}

\column{0.25\textwidth}
\textbf{Administradores (DBA)}
\begin{itemize}
    \item Gestión centralizada
    \item Seguridad y eficiencia
    \item \textbf{Rol:} 
    Mantenimiento global
\end{itemize}
\end{columns}

\begin{block}{Interacción jerárquica}
Cada tipo de usuario opera en diferentes niveles de abstracción
\end{block}
\end{frame}

\section{4. Administrador de Bases de Datos (DBA)}
\begin{frame}{Funciones Clave del DBA}
\begin{columns}[T]
\column{0.5\textwidth}
\textbf{Gestión de Esquemas}
\begin{itemize}
    \item Definición inicial (DDL)
    \item Modificación y evolución
    \item Optimización estructura
\end{itemize}

\textbf{Seguridad y Acceso}
\begin{itemize}
    \item Control de autorizaciones
    \item Gestión de permisos
    \item Políticas de seguridad
\end{itemize}

\column{0.5\textwidth}
\textbf{Almacenamiento Físico}
\begin{itemize}
    \item Selección tecnología (SSD/HDD)
    \item Diseño de índices
    \item Organización datos
\end{itemize}

\textbf{Mantenimiento}
\begin{itemize}
    \item \alert{Respaldos periódicos}
    \item Gestión de espacio
    \item Monitoreo rendimiento
\end{itemize}
\end{columns}

\begin{exampleblock}{Ejemplo práctico}
Optimizar organización física cuando aumentan volúmenes de datos
\end{exampleblock}
\end{frame}

\begin{frame}{Responsabilidades Operativas del DBA}
\begin{table}
\centering
\begin{tabular}{p{5cm}p{5cm}}
\toprule
\textbf{Función} & \textbf{Acciones Concretas} \\
\midrule
Definición del esquema & Crear tablas, relaciones y restricciones con DDL \\
\hline
Optimización física & Seleccionar almacenamiento (SSD para velocidad, HDD para costo) \\
\hline
Control de acceso & Asignar permisos (ej: finanzas puede ver ventas pero no modificar clientes) \\
\hline
Mantenimiento preventivo & \begin{tabular}{@{}l@{}}• Respaldos remotos\\• Monitoreo consultas\\• Limpieza espacio\end{tabular} \\
\hline
Solución de problemas & Optimizar consultas costosas y prevenir bloqueos \\
\bottomrule
\end{tabular}
\end{table}
\end{frame}

\section{5. Conclusión}
\begin{frame}{Síntesis Final}
\begin{columns}[T]
\column{0.5\textwidth}
\textbf{Arquitecturas}
\begin{itemize}
    \item Centralizada $\rightarrow$ Paralela $\rightarrow$ Distribuida
    \item 2 niveles $\rightarrow$ 3 niveles
    \item Transacciones: Atomicidad + Durabilidad
\end{itemize}

\textbf{Usuarios}
\begin{itemize}
    \item 4 perfiles con distintos niveles de acceso
    \item Interacción jerárquica
\end{itemize}

\column{0.5\textwidth}
\textbf{DBA}
\begin{itemize}
    \item Rol crítico en gestión
    \item Responsable de:
    \begin{itemize}
        \item Seguridad
        \item Rendimiento
        \item Disponibilidad
    \end{itemize}
    \item Funciones técnicas y operativas
\end{itemize}
\end{columns}

\vspace{5mm}
\begin{block}{Visión integral}
La arquitectura, usuarios y administración forman un ecosistema interdependiente
\end{block}

\footnotesize\textbf{Fuente:} Fundamentos De Bases De Datos (5a. Ed.) - Silberschatz
\end{frame}

\end{document}